%\documentclass[handout, hyperref={unicode}]{beamer}
\documentclass[hyperref={unicode}]{beamer}
\usetheme{Warsaw}

\usepackage{pdfpages}
\usepackage{kotex}
\usepackage{graphics}
\usepackage{graphicx}
\usepackage{hyperref}
\usepackage{colortbl}
\usepackage{amsmath}
\usepackage{amssymb}
\usepackage{amsfonts}
\usepackage[normalem]{ulem} % either use this (simple) or
\usepackage{soul} % use this (many fancier options)
\usepackage{cprotect}

\graphicspath{{./images/}}

\title{Netlogo 워크숍}
\author{이남형\inst{1}}
\institute{\inst{1} 연세대학교 경영연구소}
\date{2017.10.27.}

\begin{document}

\begin{frame}[plain]
\titlepage	
\end{frame}


\begin{frame}
\tableofcontents	
\end{frame}

\section{서론}
\subsection*{강사/연구 모임 소개}
\begin{frame}
\begin{itemize}
\item 이남형
	\begin{itemize}
	\item 경제학 박사: 오픈 소스 소프트웨어 개발을 공공재의 사적 공급으로 접근
	\item 연구 관심: 제도와 기술 변화의 역사, 게임 이론, 행위자 기반 모형
	\end{itemize}
\item 경제실험연구회
	\begin{itemize}
	\item 2014년 10월 24일 첫 세미나, 이후 54회 세미나 진행
	\item 고려대학교 경제학과 대학원생 주축
	\item 세미나 주제: 컴퓨터 기반 실험, 인간 행동 실험 $\rightarrow$ 과학 기술, 노동 시장, 환경 문제, $\cdots$
	\item ``행위자 기반 시뮬레이션을 통한 공공 연구개발비 지출제도 비교 연구," \textit{산업혁신연구}, 32(2), 227--255.
	\item 이번 학기 세미나 주제: Economics of AI
	\end{itemize}
\end{itemize}	
\end{frame}


\subsection*{워크숍 개요}
\begin{frame}{개요}
\begin{itemize}
\item 목표
	\begin{itemize}
	\item 행위자 기반 모형의 기본적인 특성을 이해하고, Netlogo 프로그래밍의 기초를 습득함
	\end{itemize}
\item 교재 및 일정
	\begin{itemize}
	\item 10월 27일, Steven F. Railsback and Volker Grimm (2012), \textit{
		Agent-Based and Individual-Based Modelling: A Practical Introduction}.
		\begin{itemize}
		\item 행위자 기반 모형은 무엇인가: Ch. 1.
		\item Netlogo 기초: Ch. 2.
		\item 나비 모형: Ch. 3--5.
		\end{itemize}
	\item 11월 3일, Lynne Hamill and Nigel Gilbert (2016), \textit{Agent‐Based Modelling in Economics}.
		\begin{itemize}
		\item 쇼핑 모형: Ch. 2.
		\end{itemize}
	\item 11월 10일, Hamill and Gilbert (2016)
		\begin{itemize}
		\item 신기술 확산(전화 보급) 모형: Ch. 4.
		\end{itemize}
	\end{itemize}
\item 장소 및 시간
	\begin{itemize}
	\item 연세대학교 경영관 B225, 오후 4시 - 오후 6시
	\end{itemize}	
\end{itemize}	
\end{frame}

\subsection*{행위자 기반 모형}
\subsubsection*{행위자는 무엇인가?}
\begin{frame}{행위자는 무엇인가?}
\begin{itemize}
\item \cite{Page:2008aa}
	\begin{quote}
	Agent-based models consist not of real people but of computational objects that interact according to rules.
	\end{quote}
%	\begin{itemize}
%	\item 대표적인 예: self-interested and rational economic agent.
%	\end{itemize}
\item 행위자
	\begin{itemize}
	\item 조직, 인간, 제도 등 어떤 목적을 갖고 있기만 하면 됨
	\end{itemize}
\end{itemize}	
\end{frame}

\subsubsection*{모형은 무엇인가?}
\begin{frame}{모형은 무엇인가?}
\begin{itemize}
\item \cite{Starfield:1990aa}
	\begin{quote}
	A model is a purposeful representation of some real system.
	\end{quote}
\item 모형의 사용 목적
	\begin{itemize}
	\item 사물이 어떻게 작동하는 지 이해하기 위해;
	\item 관찰된 패턴을 설명하기 위해;
	\item 변화에 시스템의 변화가 어떻게 반응하는 지 예측하기 위해
	\end{itemize}
\end{itemize}
\end{frame}

\begin{frame}{모형은 현실의 어디까지를 반영해야 하는가?}
	\begin{figure}[ht]
		\source{출처: \url{https://www.flickr.com/photos/26629915@N03/36879851166/}}
		\includegraphics[width=0.6\linewidth]{36879851166_2cf45baa72_o.jpg}
		\label{fig:model}
	\end{figure}
%	\begin{quote}
%	제국의 지도학은 너무 완벽해 한 지역의 지방이 도시 하나의 크기였고, 제국의 지도는 한 지방의 크기에 달했다. 하지만 이 터무니없는 지도에도 만족 못한 지도제작 길드는 정확히 제국의 크기만 한 제국전도를 만들었는데, 그 안의 모든 세부는 현실의 지점에 대응했다. 지도학에 별 관심이 없었던 후세대는 이 방대한 지도가 쓸모없음을 깨닫고, 불손하게 그것을 태양과 겨울의 혹독함에 내맡겨버렸다. 서부의 사막에는 지금도 누더기가 된 그 지도가 남아 있어, 동물과 거지들이 그 안에 살고 있다. 온 나라에 지리학 분과의 다른 유물은 남아 있지 않다. (Jorge Luis Borges, \textit{On Exactitude in Science})
%	\end{quote}		
\end{frame}

\begin{frame}
\begin{itemize}
\item 모형의 구체적 목적에 따라
	\begin{itemize}
	\item 모형을 사용하여 우리가 궁금해하는 질문에 대한 답 $\rightarrow$ 필터로 작동
		\begin{itemize}
		\item 예) 지도
		\end{itemize}
	\item 즉, 질문에 답하는 데 상관이 없거나 중요성이 충분하지 않은 현실의 특징은 제외시켜야 함
	\end{itemize}
\end{itemize}
\end{frame}

\begin{frame}
\begin{itemize}
\item 모형을 정식화 하기 위해 기초적인 아이디어가 필요
	\begin{itemize}
	\item 어떻게 시스템이 작동하는가에 대한 경험적 지식
	\item 또는 유사한 질문을 다룬 기존 모형
	\item 또는 이론
	\item 아니면 상상
	\end{itemize}
\item 그리고 우리는 모형이 실재의 세계를 얼마나 잘 반영하고 있는 지 평가할 기준이 필요
	\begin{itemize}
	\item 기준은 실재 시스템을 식별하고 특징 지울 수 있는 패턴이나 규칙성에 기반해야 함
	\end{itemize}
\item 그리고 모형에 대한 평가 이후 모형을 수정 $\rightarrow$ 재평가 $\rightarrow$ 재수정 $\cdots$
\end{itemize}	
\end{frame}

\subsection*{행위자 기반 모형은 무엇이 다른가?}
\begin{frame}{행위자 기반 모형은 무엇이 다른가?}
\begin{itemize}
\item 행위자 기반 모형
	\begin{itemize}
	\item 계산 가능성의 확장
		\begin{itemize}
		\item 고유성/이질성 부여 가능
		\item 행위자의 크기, 위치, 보유 자원, 과거 기록 등이 각기 다르게 함
		\end{itemize}
	\item (국지적) 상호작용
		\begin{itemize}
		\item 행위자는 다른 행위자와 상호작용하지만, 그들의 이웃하고만 상호작용
		\end{itemize}
	\item 행위 규칙
		\begin{itemize}
		\item 모형의 목적에 따라 결정
		\end{itemize}
	\end{itemize}	
\end{itemize}	
\end{frame}


\begin{frame}{행위자 기반 모형의 장점}
\begin{itemize}
\item 장점
	\begin{itemize}
	\item 적응적 행동
		\begin{itemize}
		\item 행위자는 그 자신, 다른 행위자 그리고 환경의 현재 상태에 행동을 적응시켜 간다.
		\end{itemize}
	\item 창발
		\begin{itemize}
		\item 시스템의 개별 구성 요소가 상호 작용 및 환경에 반응하면서 시스템 동학을 만든다.
		\end{itemize}
	\item 분석 층위를 넘나듬
		\begin{itemize}
		\item 시스템의 개체로 인해 시스템 차원의 사건이 발생
		\item 시스템 차원의 사건으로 인해 개체도 영향을 받음
		\end{itemize}
	\end{itemize}
\end{itemize}	
\end{frame}

\begin{frame}{행위자 기반 모형의 단점}
\begin{itemize}
\item 단점
	\begin{itemize}
	\item 체계적인 연구 방법론을 구성하는 중
		\begin{itemize}
		\item 연구자마다 행위자 기반 모형의 정의가 다르고,
		\item 사용하는 프로그래밍 언어도 다름.
		\item 하지만, 공통의 기반을 만들어 가는 중, 예) ODD
		\end{itemize}
	\item 창발성을 설명하는 문제
		\begin{itemize}
		\item 창발의 이유를 정확히 설명할 수 없음
		\item 만약 방정식 체계의 해로 창발을 설명할 수 있다면 행위자 기반 모형은 불필요
		\item 따라서 행위자 기반 모형으로 창발을 설명하는 것은 물건을 더듬 더듬 만지는 것과 비슷한 작업
		\end{itemize}
	\item 모형을 현실에 적용하는 문제(validation)
		\begin{itemize}
		\item 모형이 실재의 세계를 얼마나 잘 반영하는지 평가할 기준은 실재 시스템의 패턴이나 규칙성이라고 했음
		\item 앞의 타당성 문제와 결합하여
		\item 모형의 예측력이 현실에서도 작동할 수 있다고 선험적으로 말할 수 있는 근거는?
		\end{itemize}
	\end{itemize}
\end{itemize}	
\end{frame}


\subsection*{행위자 기반 모형의 연구 사례}
\begin{frame}{행위자 기반 모형 대표 연구 및 온라인 사이트}
\begin{itemize}
\item 대표 연구(경영)
	\begin{itemize}
	\item \cite{Bonabeau:2002aa}
	\item \cite{Rand:2011aa}
	\end{itemize}
\item 행위자 기반 모형 소개(다소 오래됨)
	\begin{itemize}
	\item \url{http://www2.econ.iastate.edu/tesfatsi/abusiness.htm}
	\end{itemize}
\item 온라인 강좌
	\begin{itemize}
	\item \url{https://www.complexityexplorer.org}
	\end{itemize}
\item 모형 소스코드
	\begin{itemize}
	\item \url{https://www.openabm.org}
	\item \url{http://modelingcommons.org/account/login}
	\end{itemize}
\end{itemize}	
\end{frame}

\section{Netlogo 시작하기}
\subsection*{Netlogo 설치 및 매뉴얼}
\begin{frame}{Netlogo 훑어보기}
\begin{itemize}
\item Netlogo 설치
	\begin{itemize}
	\item \url{https://ccl.northwestern.edu/netlogo/download.shtml}
	\end{itemize}
\item 매뉴얼과 친해지기
	\begin{itemize}
	\item \url{https://ccl.northwestern.edu/netlogo/docs/}
	\end{itemize}
\end{itemize}
\end{frame}

\subsection*{Netlogo 기초 구조}
\begin{frame}{행위자와 변수}
\begin{itemize}
\item 행위자
	\begin{itemize}
	\item Mobile agents $\rightarrow$ turtles.
	\item Patches $\rightarrow$ the space (the square cells).
	\item Links $\rightarrow$ connecting turtles.
	\item The observer $\rightarrow$ an overall controller of a model.
	\end{itemize}
\item 변수
	\begin{itemize}
	\item 행위자별 내장 변수
		\begin{itemize}
		\item Turtles: breed, color, heading, hidden? $\cdots$.
		\item Patches: pcolor, plabel, $\cdots$.
		\item Links: breed, color, end1, end2, $\cdots$.
		\end{itemize}
	\item 각 행위자별로 변수를 추가할 수 있음
	\item the observer 변수는 자동으로 전역 변수(global variables) $\rightarrow$ 모든 행위자가 읽거나 쓸 수 있는 변수.
	\end{itemize}
\end{itemize}	
\end{frame}

\begin{frame}
\begin{itemize}
\item 행위자는 자신의 타입이 사용할 수 있는 변수에만 접근할 수 있음
	\begin{itemize}
	\item 즉, turtles의 변수에 patches가 접근할 수 없음
	\end{itemize}
\item 예외
	\begin{itemize}
	\item observer 변수는 전역 변수로서 모든 agent가 사용 가능
	\item turtles은 자신이 현재 올라서 있는 patch의 변수를 자동으로 사용 가능
	\end{itemize}
\end{itemize}
\end{frame}

\begin{frame}{A Primitive}
\begin{itemize}
\item A Primitve
	\begin{itemize}
	\item 행위자가 할 것을 지시하는 내장된 절차 또는 명령 $\rightarrow$ 노가다를 줄이려면 사전을 열심히 봐둘 것
		\begin{itemize}
		\item commands: 행위자에게 행동을 지시
		\item reporters: 값을 계산하고 이를 사용할 수 있도록 보고 
		\end{itemize}	
	\item 맥락이 있음
		\begin{itemize}
		\item a turtle, patch, link, or observer 가 사용할 수 있는 primitive가 정해져 있음 
		\item uphill은 turtle 만; dictionary에 icon으로 표시됨
		\item 오류 메세지: ``using a patch command in a turtle context."
		\end{itemize}
	\end{itemize}
\end{itemize}	
\end{frame}

\section{나비모형: ODD 프로토콜}
\subsection*{ODD 프로토콜}
\begin{frame}[fragile]{ODD 프로토콜}
\begin{itemize}
\item 많은 경우 ABM을 reimplement 하거나 결과를 replicate하기 어려움
\item 모델에 대한 묘사가 사실 설명, 정당화, 기타 등등 것들에 대한 논의를 다 담아야 하기 때문
\item 손 쉽게 이해할 수 있도록 표준화된 설명 방식을 만들면 되지 않을까? $\rightarrow$ 프로토콜
\item \verb|ODD_protocol.pdf/xls| 참고 
\end{itemize}	
\end{frame}

\subsection*{나비 모형과 ODD 프로토콜}
\begin{frame}{나비 모형과 ODD 프로토콜}
\begin{itemize}
\item 목적
	\begin{itemize}
	\item 나비의 언덕 오르기 행위와 지형도의 상호 작용으로 통로가 만들어지는 조건은 무엇인가?
	\item 나비의 언덕 오르기에 영향을 미치는 다양한 요소가 통로의 창발에 영향을 미치는가?
	\end{itemize}
\end{itemize}	
\end{frame}

\begin{frame}
\begin{itemize}
\item 독립체, 상태 변수, 비례
	\begin{itemize}
	\item 나비
		\begin{itemize}
		\item 위치가 중요 특징 $\rightarrow$ patch의 중앙에 위치 시킴
		\end{itemize}
	\item patch
		\begin{itemize}
		\item 150 $\times$ 150 
		\item patch는 하나의 변수만 가짐: elevation
		\end{itemize}
	\item 1,000 steps 시행, a patch는 25 $\times$ 25$m^{2}$에 대응
	\end{itemize}
\end{itemize}	
\end{frame}

\begin{frame}
\begin{itemize}
\item 전체 과정 및 스케쥴 
	\begin{itemize}
	\item 과정: 단일. 나비의 움직임
	\item 매 step마다 나비가 한 번 움직임
	\item 기본 모형에서는 상호작용이 없으므로, 나비의 움직임 순서는 중요하지 않음 
	\end{itemize}
\end{itemize}	
\end{frame}

\begin{frame}
\begin{itemize}
\item 설계 개념
	\begin{itemize}
	\item 관심 변수: 통로를 어떻게 정의할 것인가?
	\item 통로는 모형의 두 가지 요소로부터 창발
		\begin{itemize}
		\item 나비의 적응적 움직임: 즉, 언덕 오르기
		\item 나비가 움직이는 지형
		\end{itemize}
	\item 나비는 자신의 주변에서 가장 높은 곳을 파악할 수 있어야 함 $\rightarrow$ 얼마나 먼거리에 있는 patch의 높이까지 알 수 있는가로 모형을 변화시킴
	\item 모형에는 상호 작용이 없음
	\item 확률 과정
		\begin{itemize}
		\item 현실에서 항상 언덕 오르기를 할 수 있는 것은 아님
		\item 가장 높은 지역을 찾는 능력에 제한이 있고, 지형 이외에 다른 요소, 이를 테면 꽃을 따라 간다든지
		\item 이를 묘사하기 위해 단순한 파라미터 $q$(언덕을 곧장 오를 확률)를 도입
		\end{itemize}
	\end{itemize}
\end{itemize}
\end{frame}

\begin{frame}
\begin{itemize}
\item 초기화
	\begin{itemize}
	\item 지형도
		\begin{itemize}
		\item 단순 가상 공간에서
		\item 실제 공간으로
		\end{itemize}
	\item 나비 500마리  $\rightarrow$ 하나의 patch 또는 좁은 지역에 위치
	\end{itemize}
\item 입력 자료
	\begin{itemize}
	\item 환경 변화 없음 $\rightarrow$ 입력 자료 없음
	\end{itemize}
\item 하위 모델
	\begin{itemize}
	\item 나비가 어떻게 언덕을 오르는가? 곧장 또는 무작위로? 
	\end{itemize}
\end{itemize}
\end{frame}

\section{나비모형: Netlogo}
\subsection*{언덕 오르기: ODD로부터 Netlogo로}
\begin{frame}[fragile]{ODD로부터 Netlogo로}
\begin{itemize}
\item 새로운 파일: \verb|Butterfly-1.nlogo|
\item 목적
	\begin{itemize}
	\item \url{http://www.railsback-grimm-abm-book.com/Chapter04/ButterflyModelODD.txt}
	\item \verb|Info| 탭에 작성
	\end{itemize}	
\end{itemize}	
\end{frame}

\begin{frame}[fragile]
\begin{itemize}
\item 독립체, 상태 변수, 비례
		\begin{verbatim}
		globals
		[
		]

		patches-own
		[
		]

		turtles-own
		[
		]
		\end{verbatim}	
	\begin{itemize}
	\item \verb|Check| 클릭 $\rightarrow$ 아무 문제 없어야 정상
	\item ODD를 보면, 
		\begin{itemize}
		\item turtles의 상태 변수는 위치 $\rightarrow$ 따로 정할 필요 없음
		\item patches의 상태 변수 elevation $\rightarrow$ 정해야 함
		\end{itemize}
	\end{itemize}
\end{itemize}	
\end{frame}

\begin{frame}[fragile]
	\begin{verbatim}
	patches-own
	[
	elevation
	]
	\end{verbatim}
\begin{itemize}
\item now, move on.
\item 모형의 공간
	\begin{itemize}
	\item \verb|Interface| tab $\rightarrow$ \verb|Settings|.
	\item \verb|Location of origin| $\rightarrow$  \verb|Corner| and \verb|Bottom Left|.
	\item \verb|max-pxcor| 와 \verb|max-pycor| $\rightarrow$ 149
	\item \verb|World wraps horizontally|와 \verb|World wraps vertically| 체크 해제.
	\item World 창이 너무 크게 보일 것, 수정해봅시다.
		\begin{itemize}
		\item \verb|Interface| tab $\rightarrow$ \verb|Settings|.
		\item \verb|Patch size| $\rightarrow$ 3.
		\end{itemize}
	\end{itemize}
\end{itemize}	
\end{frame}

\begin{frame}[fragile]
\begin{itemize}
\item 초기화
\end{itemize}
	\begin{verbatim}
	turtles-own
	[
	]

	to setup
	  ca
	  ask patches
	  [

	  ]
	  reset-ticks
	end
	\end{verbatim}
\end{frame}

\begin{frame}[fragile]
\begin{itemize}
\item 지형 생성
\end{itemize}
	\begin{verbatim}
	  ask patches
	  [
	   let elev1 100 - distancexy 30 30
	   let elev2 50 - distancexy 120 100

	   ifelse elev1 > elev2
	    [set elevation elev1]
	    [set elevation elev2]

	   set pcolor scale-color green elevation 0 100
	  ]
	\end{verbatim}
	\begin{itemize}
	\item[Note] Netlogo의 명령어는 의미별로 모두 띄어쓸 것 
	\end{itemize}
\end{frame}

\begin{frame}[fragile]
\begin{itemize}
\item 지형 설명
	\begin{itemize}
	\item $(30,30)$과 $(120, 100)$에 정상이 있는 두 개의 언덕
	\item 첫번째 언덕의 정상 높이는 100 units $\rightarrow$ 1 unit $=$ 1 m
	\item 두번째 언덕의 정상 높이는 50 units
	\item patch의 높이는 위의 두 정상으로부터 수평 거리가 한 단위씩 멀어질 수록 1 unit 씩 감소
	\item \verb|ifelse| $\rightarrow$ 두 개의 elevation 중 항상 높은 값을 갖도록 
	\item \verb|scale-color| $\rightarrow$ elevation에 따라 음영 처리
	\end{itemize}
\item 지도를 잘 그렸나 확인
	\begin{itemize}
	\item \verb|Interface| 탭 $\rightarrow$ \verb|Button| $\rightarrow$ \verb|Button| $\rightarrow$ ``setup"
	\end{itemize}
\end{itemize}
\end{frame}

\begin{frame}[fragile]
\begin{itemize}
\item 움직이는 행위자 turtle을 만들자.
	\begin{verbatim}
	 set pcolor scale-color green elevation 0 100
	]
	
	crt 1
	[
	 set size 2
	 setxy 85 95
	]
	\end{verbatim}
\item 잘 만들어졌나 확인
\end{itemize}	
\end{frame}

\begin{frame}[fragile]
\begin{itemize}
\item 스케쥴 $\rightarrow$ 나비(turtle)가 1 step에 1번 움직임 
		\begin{verbatim}
		  reset-ticks
		end
		
		to go
		  ask turtles [move]
		end

		to move

		end
		\end{verbatim}		
\end{itemize}	
\end{frame}

\begin{frame}[fragile]
\begin{itemize}
\item $\rightarrow$ 1000 steps
		\begin{verbatim}
		to go
		  ask turtles [move]
		  tick
		  if ticks >= 1000 [stop]
		end
		\end{verbatim}
\item \verb|Interface|에 \verb|go| \verb|button| 추가 
	\begin{itemize}
	\item \verb|Interface|에서 구성을 움직이거나, 리사이즈, 편집하고 싶다면, 
	\item 구성 요소를 선택하고 마우스 오른쪽 클릭 후, ``Select" 선택
	\end{itemize}
\item \verb|go| 눌러 봅시다 $\rightarrow$ 아무 일도 없어야 정상
\end{itemize}	
\end{frame}

\begin{frame}[fragile]
\begin{itemize}
\item \verb|to move|의 내용을 채우지 않았음
\item $q$의 확률로 곧장 언덕을 오르고, $1-q$의 확률로 무작위 이동 $\rightarrow$ 국지적인 정상에 오를 때까지
	\begin{verbatim}
	to move
	  ifelse random-float 1 < q
	   [uphill elevation]
	   [move-to one-of neighbors]
	end
	\end{verbatim}
\end{itemize}	
\end{frame}

\begin{frame}[fragile]
\begin{itemize}
\item \verb|to move|
	\begin{itemize}
	\item \verb|random-float NUMBER| $\rightarrow$ manual 참고
	\item \verb|uphill|, \verb|move-to one-of neighbors|: 모두 primitives
	\end{itemize}
\item \verb|check| or \verb|go| $\rightarrow$ 에러!, $q$ 정의 필요
		\begin{verbatim}
		globals
		[
		  q
		]
		...
		to setup
		...
		  set q 0.4
		end
		\end{verbatim}
\end{itemize}	
\end{frame}

\begin{frame}[fragile]
\begin{itemize}
\item 나비들은 잘 가고 있을까?
	\begin{verbatim}
	  crt 1
	  [
	    set size 2
	    setxy 85 95
	    pen-down
	  ]
	\end{verbatim}
\item 한번에 보내고 싶다면
	\begin{itemize}
	\item \verb|Interface| 탭, \verb|go| 버튼, 오른쪽 클릭, \verb|forever| 박스 체크
	\end{itemize}	
\end{itemize}	
\end{frame}

\section{나비모형: 과학 연구를 향하여}
\subsection*{서론과 목표}
\begin{frame}{논문에는 안 나오는 것}
\begin{itemize}
\item 진짜 일은 최초의 모델이 만들어진 이후에 시작
	\begin{itemize}
	\item 모형 분석과 세부 수정의 반복 작업
	\item $\rightarrow$ 모형을 갖고 놀기 (Playing) 
	\item $\rightarrow$ ``What would happen if $\ldots$"의 마인드가 중요
	\end{itemize}
\item NetLogo 모델 라이브러리
	\begin{itemize}
	\item 실현에는 장점 $\rightarrow$ 그들이 세운 가정과 수식이 어떻게 작동하는지 보여줌
	\item 그러나 이 것을 가지고 어떻게 과학을 할지 알려주지 않음
		\begin{itemize}
		\item 하지만 어떻게 아이디어와 개념을 만들었고, 
		\item 가설을 발전시키고 테스트 했는지, 
		\item 관찰된 형상을 얼마나 좁게 또는 얼마나 일반화시켜서 설명할 것인지 등을 보여주지 않음
		\end{itemize}
	\end{itemize}
\end{itemize}	
\end{frame}

\begin{frame}{나비 모형으로 과학하기}
\begin{itemize}
\item 나비 모형의 목적: 통로
	\begin{itemize}
	\item 즉, 나비를 유인하는 어떤 요소가 없는 데도, 나비가 움직이는 경로가 집중된다는 것
	\item 그리고 이 것이 어떻게 언제 나타나는 지는 설명해야함
		\begin{itemize}
		\item 하지만 현재의 모델은 나비의 움직임만 모델링
		\end{itemize}
	\item 가상 지형을 실제 지형으로 대체 
	\end{itemize}
\end{itemize}	
\end{frame}

\subsection*{통로의 관찰}
\begin{frame}
\begin{itemize}
\item 무엇을 할 지부터 생각해야 함
\item 통로의 정의가 필요
	\begin{itemize}
	\item 우리 모델에서 나비는 아무데서나 출발하고 멈출 수 있음
	\item 나비가 국지적 정상에 도달하면 멈춘다고 가정
		\begin{itemize}
		\item 국지적 정상: 주변의 8개 patch보다 높은 patch
		\end{itemize}
	\item 모든 나비가 사용하는 통로의 폭
		\begin{equation*}
						= \dfrac{\text{각각의 나비가 방문하는 patch}}{\text{출발점과 도착점을 연결하는 (평균) 직선 거리}}
		\end{equation*}
	\item 모든 나비가 같은 직선 경로를 따라 올라간다고 생각하면, 통로의 폭은 작을 것 $\rightarrow$ 다른 경로를 따른다면, 폭이 커질 것
	\end{itemize}
\item 모델의 분석을 위해 통로의 폭과 나비가 곧장 언덕을 오를 확률 $q$의 관계 그래프를 그려보자.
\end{itemize}	
\end{frame}

\begin{frame}[fragile]{프로그램의 수정}
\begin{itemize}
\item \verb|Interface| 탭, \verb|sliders| 생성 $\rightarrow$ $q$ 변경
	\begin{itemize}
	\item 0.0에서 1.0까지 0.01씩 증가하도록
	\end{itemize}
\item 경고!
	\begin{itemize}
	\item \verb|sliders|, \verb|switches|, \verb|choosers|는 모두 전역 변수를 사용
	\item 따라서 중복 $\rightarrow$ 코드에서 주석 처리
	\item $\rightarrow$ 관련 명령도 모두 주석 처리 
		\begin{itemize}
		\item \verb|set q 0.4|를 그대로 두면 $q = 0.4$로 고정
		\end{itemize}
	\end{itemize}	
\end{itemize}	
\end{frame}

\begin{frame}[fragile]{주석}
\begin{itemize}
\item \verb|;|
	\begin{itemize}
	\item 지금 하고 있는 일이 무엇인지 설명
	\item 변수 설명
	\item 각 단계의 맥락을 설명
	\item 끝이 어디인지 확인
	\item 긴 프로그램의 경우 procedures 가 어디서 끝나는 지 확인할 때
	\end{itemize}
\end{itemize}	
\end{frame}

\begin{frame}[fragile]
\begin{itemize}
\item 50 마리 생성
	\begin{verbatim}
	crt 50
	\end{verbatim}
\item local hilltop에 올라오면 멈춤
	\begin{verbatim}
	to move
       if elevation >= 
       [elevation] of max-one-of neighbors [elevation] 
       [stop]
	\end{verbatim}
\end{itemize}	
\end{frame}

\begin{frame}[fragile]
\begin{itemize}
\item corridor width를 계산해보자
	\begin{itemize}
	\item 나비가 들른 patch의 수
		\begin{verbatim}
		patches-own
		  [
		  used?
		  ]
		\end{verbatim}
		\begin{itemize}
		\item[Note] boolean (true-false) 변수: 변수명 끝에 $?$
		\end{itemize}
	\item 출발점과 도착점의 평균 거리
		\begin{verbatim}
		turtles-own
		[
		  start-patch
		]
		\end{verbatim}
	\end{itemize}
\end{itemize}	
\end{frame}

\begin{frame}[fragile]
\begin{itemize}
\item 새로 만든 변수를 초기화 해주어야 함
	\begin{verbatim}
	  ask patches
	    [
	    ...
	     set used? false
	    ]

	  crt 50
	    [
	     ...
	     set start-patch patch-here
	    ]
	\end{verbatim}
		\begin{itemize}
		\item[Note] initializing variables: 새로운 변수는 다른 값을 지정하기 전에는 $0$에서 시작. 프로그램 작동에는 문제를 일으키지 않지만, 분석에서는 문제가 될 수 있는 데, 알기도 찾기도 어려운 error $\rightarrow$ 값을 주고 시작하는 습관을 들이자.
		\end{itemize}
\end{itemize}	
\end{frame}

\begin{frame}[fragile]
\begin{itemize}
\item 나비가 지나간 patch를 계산해야 함
	\begin{verbatim}
	to move
	 ...
	   set used? true
     end
	\end{verbatim}
\item 이제 corridor width 를 계산해야 함
	\begin{itemize}
	\item local 변수: \verb|final-corridor-width|, 
	\item 보고문 \verb|corridor-width|
		\begin{verbatim}
		to go
		 ...
		 if ticks >= 1000 [stop]
				  
		 let final-corridor-width corridor-width
		end
		 ...
		to-report corridor-width
		end
		\end{verbatim}
	\end{itemize}	
\end{itemize}	
\end{frame}


\begin{frame}[fragile]
\begin{verbatim}
to-report corridor-width
  let num-of-visited-patches count patches 
  with [used? = true]
	  
  ask turtles
  [
  set the-distance distance start-patch
  ]
      
  set mean-of-the-distance mean [the-distance] 
  of turtles  
  report num-of-visited-patches / mean-of-the-distance
end
\end{verbatim}
\end{frame}

\begin{frame}[fragile]
\begin{verbatim}
globals
[
; q
mean-of-the-distance
]

turtles-own
[
 start-patch
 the-distance
]
\end{verbatim}	
\end{frame}

\begin{frame}[fragile]
\begin{itemize}
\item 결과 관찰하기
	\begin{itemize}
	\item 그래프 그리기
			\begin{verbatim}
			to go
			  ask turtles [move]
			  plot corridor-width
			  ...
			end
			\end{verbatim}
		\begin{itemize}
		\item \verb|Interface| 탭에 그래프 창 그리고
		\item 그래프 이름은 \verb|Corridor Width|
		\item \verb|Update commands|의 내용은 모두 삭제
		\end{itemize}
	\item csv 파일로 내보내기
		\begin{verbatim}
		to go
		 ...
		 export-plot "Corridor width"
		 (word "Corridor-output-for-q-" q ".csv")	
		end
	\end{verbatim}
	\end{itemize}
\end{itemize}
\end{frame}

\begin{frame}[fragile]
\begin{itemize}
\item UTM 같은 좌표 시스템을 바로 NetLogo에서 사용할 수 없음 $\rightarrow$ 전환 필요. 관련 소프트웨어에서 알아서
	\begin{itemize}
	\item 가로로 한 줄에, grid point or cell 좌표 정보, 공간 특성 정보가 들어가도록 작성
	\item NetLogo의 patch 하나의 크기($1 \times 1$)에 맞춰 실제 공간의 scale을 보정 해주어야 함
	\end{itemize}
\item NetLogo 모델에서 공간 설정
	\begin{itemize}
	\item \verb|Settings| 창 또는
	\item 명령어 \verb|resize-world|
	\end{itemize}	
\end{itemize}	
\end{frame}

\begin{frame}[fragile]
\begin{itemize}
\item 공간 파일을 불러옵시다.
	\begin{itemize}
	\item 파일 버전 변경
	\item 불러올 공간 파일을 같은 폴더에 둠. 아니면 \verb|user-file| 명령어 사용
		\begin{verbatim}
		to setup
		...
		  file-open "ElevationData.txt"
		  while [not file-at-end?]
		  [
		    let next-x file-read
		    let next-y file-read
		    let next-elevation file-read
		    ask patch next-x next-y [set elevation next-elevation]
		  ]
		  file-close
		end
		\end{verbatim}
	\end{itemize}		
\end{itemize}	
\end{frame}

\begin{frame}[fragile]
\begin{itemize}
\item \verb|max|, \verb|min|을 써서 최고 높이와 최저 높이에 따라 색을 변화
	\begin{verbatim}
	to setup
	...
	file-close
		
	ask patches
	 [  
	  set pcolor scale-color green elevation min [elevation] of patches max [elevation] of patches
	  set used? false
	 ]
	end
	\end{verbatim}
\end{itemize}	
\end{frame}

\subsection*{반복 실험과 BehaviorSpace}
\begin{frame}{BehaviorSpace}
\begin{itemize}
\item 서로 다른 시나리오가 반복되는 실험(experiments)을 해야 함
	\begin{itemize}
	\item 모델과 변수/입력 자료/초기 조건의 한 집합이 하나의 시나리오
	\item 만약 확률 과정이 없다면, 하나의 시나리오를 여러 번 반복하더라도 동일한 결과가 나올 것
	\item 반복 실험은 모델에서 확률 과정만 변화하고 다른 요소는 그대로 두고 실행하는 것
		\begin{itemize}
		\item 무작위로 생성되는 수가 바뀌거나
		\item 입력 자료를 바꾸거나
		\item 초기 값을 바꾸는 것도 다른 시나리오가 될 수 있음
		\end{itemize}
	\end{itemize}
\end{itemize}
\end{frame}

\begin{frame}
\begin{itemize}
\item BehaviorSpace가 우리를 구원하리라.
	\begin{itemize}
	\item NetLogo User Manual $\rightarrow$ Features $\rightarrow$ BehaviorSpace Guide.
	\item 전역 변수 값을 바꿔서 시나리오를 생성
	\item 각 시나리오를 반복 시행 (repetitions)
	\item 각 시행의 결과를 모아서 하나의 파일로 작성
	\end{itemize}
\item 우리의 경우, $q$
\end{itemize}	
\end{frame}

\bibliographystyle{apalike}
\bibliography{my_library}

\end{document}
