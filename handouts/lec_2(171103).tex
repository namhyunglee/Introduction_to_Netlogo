%\documentclass[handout, hyperref={unicode}]{beamer}
\documentclass[hyperref={unicode}]{beamer}
\usetheme{Warsaw}

\usepackage{pdfpages}
\usepackage{kotex}
\usepackage{graphics}
\usepackage{graphicx}
\usepackage{hyperref}
\usepackage{colortbl}
\usepackage{amsmath}
\usepackage{amssymb}
\usepackage{amsfonts}
\usepackage[normalem]{ulem} % either use this (simple) or
\usepackage{soul} % use this (many fancier options)
\usepackage{cprotect}

\graphicspath{{./images/}}

\title{Netlogo 워크숍}
\author{이남형\inst{1}}
\institute{\inst{1} 연세대학교 경영연구소}
\date{\today}

\begin{document}

\begin{frame}[plain]
\titlepage	
\end{frame}

\begin{frame}
\tableofcontents	
\end{frame}

\begin{frame}{이번 강의의 목표}
\begin{itemize}
\item 목록(행렬) 만들기
\item 모형 test 하기와 wishlist 만들어 보기
\end{itemize}	
\end{frame}


\section{쇼핑 모형: ODD 프로토콜}
\begin{frame}{쇼핑 모형과 ODD 프로토콜}
\begin{itemize}
\item 목적
	\begin{itemize}
	\item 과일과 야채 거래 시장을 Netlogo로 구현
	\end{itemize}
\item 독립체
	\begin{itemize}
	\item 소비자와 판매자
	\end{itemize}
\item 확률 과정
	\begin{itemize}
	\item 쇼핑 리스트의 아이템 
	\item 판매자가 보유하는 아이템, 아이템의 소매 가격
	\end{itemize}
\item 초기화
	\begin{itemize}
	\item 소비자의 수, 구매를 위한 이동 속도, 구매 대안의 수
	\item 판매자의 보유 아이템 수
	\end{itemize}
\item 결과
	\begin{itemize}
	\item 시간 변화에 따라, 구매하지 못한 아이템의 평균 개수
	\item 평균 구매 속도
	\end{itemize} 
\end{itemize}	
\end{frame}


\begin{frame}{ODD와 Pseudo Codes}
\begin{itemize}
\item Pseudo Codes $\in$ ODD
	\begin{itemize}
	\item ODD는 ABM을 이용한 연구의 전체 설계도
	\item Pseudo Codes는 ABM을 위한 프로그래밍 가이드
		\begin{itemize}
		\item ABM을 위한 프로그래밍은 반드시 Netlogo일 필요는 없음
		\item C++, JAVA, Python, Swarm, $\cdots$
		\end{itemize}
	\end{itemize}
\end{itemize}	
\end{frame}

\begin{frame}{Pseudo Codes}
\begin{itemize}
\item 판매를 위한 12개의 과일과 채소 리스트 설정
\item 12개의 과일과 채소의 도매 가격 설정 $\rightarrow$ 1--100 중 랜덤
\item 판매자
	\begin{itemize}
	\item 9개, 위치는 활동 공간의 중앙, 붉은색 집 
	\item 각각의 가게는 12개의 과일과 채소 중 랜덤하게 보유 
		\begin{itemize}
		\item 몇 개를 보유할 지 슬라이더를 이용하여 통제
		\item 소매가격은 도매 가격에 1--30\% 를 랜덤하게 마크업
		\end{itemize}
	\end{itemize}
\item 소비자
	\begin{itemize}
	\item 몇 명의 소비자를 둘지는 슬라이더로 선택할 수 있음
	\item 위치는 랜덤하게 결정, 노란색 사람
	\end{itemize}	
\end{itemize}	
\end{frame}

\begin{frame}
\begin{itemize}
\item 소비자 속성
	\begin{itemize}
	\item 1--8 종의 랜덤 아이템이 포함된 쇼핑 리스트
	\item 방문하지 않은 판매자 리스트: 초기값 9
	\item 사용 금액: 초기값 0
	\item 소비자의 쇼핑 리스트 평균 길이를 계산
	\end{itemize}
\item 쇼핑
	\begin{itemize}
	\item 가장 싸게 파는 판매자를 스캔
	\item 소비자는 선택한 판매자를 방문
		\begin{itemize}
		\item 재방문은 없음
		\item 구매 기록 저장
		\item 지불 가격 합산
		\end{itemize}
	\item 쇼핑 리스트가 비면 소비자는 집으로 돌아감	
	\end{itemize}
\item 결과를 보고하고 그래프를 그림	
\end{itemize}	
\end{frame}

\section{쇼핑 모형: Netlogo}
\subsection{기본 모형}
\begin{frame}[fragile]{Setup: 전역 변수 설정}
\begin{verbatim}
	globals 
	 [
	  fruit-and-veg
	  mean-items
	 ]
	
	
	breed [ shoppers shopper ]	 
	breed [ traders trader ]
	 
	shoppers-own [ shopping-list ] 
	traders-own [ stock ]
\end{verbatim}
\end{frame}

\begin{frame}[fragile]{Setup: 상품 목록 설정}
\begin{verbatim}
	to setup
	 clear-all
	 
	 set fruit-and-veg
	  [
	  "apples" "bananas" "oranges" "plums" "mangos" 
	  "grapes" "cabbage" "carrots" "potatos" "lettuce" 
	  "tomatoes" "beans" 
	  ]
	  
	 let xs [ -12 -9 -6 -3 0 3 6 9 12 ]
	 
	 reset-ticks	  
	end	  
\end{verbatim}	
\end{frame}

\begin{frame}[fragile]{Setup: 판매자 생성}
\begin{verbatim}
	to setup
	...
	 let xs [ -12 -9 -6 -3 0 3 6 9 12 ]
	 foreach xs [ ?1 ->
	  create-traders 1
	  [
	   set shape "house"
	   setxy ?1 0
	   set color red
	   set stock n-of n-items-stocked fruit-and-veg
	  ]
	 ]
	...  
	end	  
\end{verbatim}	
\end{frame}

\begin{frame}[fragile]{Setup: 소비자 생성}
\begin{verbatim}
	
     create-shoppers n-shoppers
     [
      set shape "person"
      setxy random-pxcor random-pycor
      set color yellow
      set shopping-list n-of (1 + random 8) fruit-and-veg 
     ]
     
     set mean-items mean [ length shopping-list ] 
     of shoppers
	 
	 reset-ticks  
	end	  
\end{verbatim}	
\end{frame}

\begin{frame}[fragile]{Setup: 시각적으로 확인하기}
\begin{itemize}
\item \verb|Interface| 탭
	\begin{itemize}
	\item \verb|Button| $\rightarrow$ \verb|Button| $\rightarrow$ setup
	\item \verb|Button| $\rightarrow$ \verb|Slider| 
		\begin{itemize}
		\item $\rightarrow$ n-items-stocked, 0 -- 1 -- 5
		\item $\rightarrow$ n-shoppers, 0 -- 1 -- 20
		\end{itemize}
	\end{itemize}
\item 셋업 확인
	\begin{itemize}
	\item \verb|setup| 버튼 누르기
	\item \verb|n-shoppers| 바꿔보기
		\begin{itemize}
		\item ok?
		\end{itemize}
	\item stock 체크하기 
		\begin{itemize}
		\item $\rightarrow$  \verb|monitor| $\rightarrow$  \verb|stock| 
		\item 에러가 정상. 에러 내용을 확인해보자.
		\item 어떻게 체크할 수 있을까?
		\item \verb|show ( word "has " stock )|
		\end{itemize}
	\end{itemize}
\end{itemize}	
\end{frame}




\begin{frame}[fragile]{Go-procedure: 소비자가 판매자에게 가기}
\begin{verbatim}
to go
ask shoppers with [ not empty? shopping-list ]
  [
    let stall one-of traders
    
    face stall
    
    while [ patch-here != [patch-here] of stall ]
       [forward 0.005 * walking-speed]
  ]
  
tick   
end

\end{verbatim}	
\end{frame}

\begin{frame}[fragile]
\begin{verbatim}

[forward 0.005 * walking-speed]

let purchases filter [ ?1 -> member? ?1 [stock] of stall ]
  shopping-list

foreach purchases
 [ ?1 -> set shopping-list remove ?1 shopping-list ] 
 if empty? shopping-list [ set ycor -16 ]
 ]

set mean-items mean [ length shopping-list ] of shoppers

if mean-items = 0 [ stop ]


tick
\end{verbatim}	
\end{frame}

\begin{frame}[fragile]
\begin{itemize}
\item \verb|Interface| 탭
	\begin{itemize}
	\item \verb|Button| $\rightarrow$ \verb|Slider| 
		\begin{itemize}
		\item $\rightarrow$ walking-speed, 0 -- 0.1 -- 0.5
		\end{itemize}
	\item \verb|Button| $\rightarrow$ \verb|monitor| $\rightarrow$ 
		\begin{itemize}
		\item \verb|mean-items|
		\item Mean number of items left to buy, 
		\end{itemize}	
	\end{itemize}
\end{itemize}	
\end{frame}
go
\begin{frame}[fragile]{기본 모형 실행}
\begin{itemize}
\item \verb|Interface| 탭
	\begin{itemize}
	\item \verb|Button| $\rightarrow$ \verb|Button| $\rightarrow$ 
		\begin{itemize}
		\item go
		\end{itemize}	
	\end{itemize}
\item 모형 구동 확인
	\begin{itemize}
	\item \verb|n-items-stocked|, \verb|n-shoppers|, \verb|walking-speed| 바꿔 가면서
	\item 끝?
	\end{itemize}
\item 가격!
\end{itemize}	
\end{frame}

\subsection{가격이 반영된 모형}
\begin{frame}{가격과 행위자}
\begin{itemize}
\item 가격 변수
	\begin{itemize}
	\item 판매자: 각 아이템별 가격
		\begin{itemize}
		\item 도매 가격으로 구입한 후 마크업(랜덤)을 붙인 소매 가격 
		\end{itemize}
	\item 소비자: 소비액
		\begin{itemize}
		\item 소비액 기록
		\item 예산은? $\rightarrow$ 추후 연구과제(wishlist)
		\end{itemize}
	\end{itemize}	
\end{itemize}	
\end{frame}


\begin{frame}[fragile]{Setup: 가격 변수 설정}
\begin{verbatim}
globals
 [
	...
  fruit-and-veg-prices
 ] 	
 
shoppers-own [ shopping-list spent ] 
traders-own [ stock prices ] 

to setup
	...
 set fruit-and-veg-prices n-values 
  (length fruit-and-veg) [ 1 + random 100 ]
  
 let xs [ -12 -9 -6 -3 0 3 6 9 12 ]
\end{verbatim}
\end{frame}

\begin{frame}[fragile]{Setup: 상점별 마크업 가격 부여}
\begin{verbatim}
 
foreach xs [ ?1 ->
  create-traders 1
  [   
;     show ( word "has " stock )
    set prices [ ]
    let mark-up ( 1 + random 30 ) / 100
    foreach stock [ ??1 -> 
     set prices lput 
     ( ( 1 + mark-up) * 
      ( item ( position ??1 fruit-and-veg ) 
       fruit-and-veg-prices ) 
     ) prices
    ]
  ]
 ]    
 
\end{verbatim}
\end{frame}


\begin{frame}[fragile]{Go: 소비액 계산}
\begin{verbatim}
to go

ask shoppers with [ ...
 
 foreach purchases [ ?1 ->
  [ 
    set spent spent + item ( position ?1 [ stock ] of 
           stall ) [ prices ] of stall 
    set shopping-list remove ?1 shopping-list
  ] 
\end{verbatim}
\end{frame}

\begin{frame}[fragile]{결과값: 소비액 관찰}
\begin{itemize}
\item \verb|Interface| 탭
	\begin{itemize}
	\item \verb|Button| $\rightarrow$ \verb|plot| $\rightarrow$ 
		\begin{itemize}
		\item Spending
		\item \verb|plot mean [ spent ] of shoppers |
		\end{itemize}	
	\end{itemize}
\end{itemize}	
\end{frame}


\begin{frame}[fragile]{가격이 반영된 모형 실행}
\begin{itemize}
\item Spending 창 관찰
\item 모형의 약점은?
	\begin{itemize}
	\item mean number of items left to buy 관찰
	\end{itemize}
\end{itemize}	
\end{frame}

\subsection{모든 판매자를 방문하고 소비를 결정하는 모형}
\begin{frame}[fragile]{판매자 방문과 선택}
\begin{itemize}
\item 소비자가 최초에는 랜덤하게 판매자 방문 $\rightarrow$ 이후 판매자를 선택
\item 만약 판매자를 모두 방문했는데 원하는 상품이 없다면 쇼핑 중지
\end{itemize}	
\end{frame}

\begin{frame}[fragile]{Setup: 방문한 판매자는?}
\begin{verbatim}
shoppers-own [ shopping-list spent not-yet-visited ]	
...
to setup
...
create-shoppers n-shoppers
[
  ...
  set not-yet-visited traders
  set shopping-list ...
]

\end{verbatim}	
\end{frame}

\begin{frame}[fragile]{Go: 방문한 판매자 기억시키기}
\begin{verbatim}
to go
	
ask shoppers with [ not empty? shopping-list ]
 [
;    let stall one-of traders
   let stall one-of not-yet-visited
   ...
   set not-yet-visited not-yet-visited with 
       [ self != stall]
   let purchases filter [ ... 
 ]

\end{verbatim}	
\end{frame}


\begin{frame}[fragile]{Go: 구매 중지 및 구매 못한 상품 기록하기}
\begin{verbatim}
to go
	
ask shoppers with [ not empty? shopping-list ]
 [
   let stall one-of not-yet-visited
   if stall = nobody
    [
     show ( word "No one sells " shopping-list )
     set shopping-list []
     stop
    ]
  ...  	
\end{verbatim}	
\end{frame}

\begin{frame}[fragile]{판매자 방문과 중지 반영 모형 실행}
\begin{itemize}
\item \verb|Interface| 탭
	\begin{itemize}
	\item \verb|go| 버튼 $\rightarrow$ 마우스 오른쪽 버튼 클릭 \verb|Edit| $\rightarrow$ 
		\begin{itemize}
		\item 박스 \verb|forever| 체크
		\end{itemize}	
	\end{itemize}
\item 끝?
	\begin{itemize}
	\item 책에서는 소비자가 일부 판매자만 방문하고, 자신이 구입하고자 하는 상품을 가장 싸게 파는 판매자로부터 구매하는 알고리듬을 구현
	\item 자신의 wishlist를 늘려가 보자
	\end{itemize}
\end{itemize}		
\end{frame}

%\subsection{제한된 합리성을 반영한 모형}
%\begin{frame}{제한된 합리성을 정의하기}
%\begin{itemize}
%\item 소비자는 일부의 판매자를 방문
%\item 자신이 구입하고자 하는 상품을 가장 싸게 파는 판매자로부터 구매
%\end{itemize}	
%\end{frame}
%
%\begin{frame}[fragile]{일부 판매자 방문}
%\begin{verbatim}
%to go
%ask shoppers ;with [ not empty? shopping-list ]
%[
%let route search-before-buying
%
%foreach route
% [ ?1 ->
%  let stall ?1  
%
%;    let stall one-of not-yet-visited
%;    if stall = nobody
%;     [
%;        show ( word "No one sells " shopping-list )
%;        set shopping-list []
%
%\end{verbatim}	
%\end{frame}
%
%\begin{frame}[fragile]
%\begin{verbatim}
%;        stop
%;     ]
%face stall
%while [ patch-here != ...
%set not-yet-visited ...
%
%let purchases buy-from-stall shopping-list stall 
%;filter [ ?1 -> member? ?1 [stock] of stall ] shopping-list
%
%foreach purchases 
% [ ...
% 
%if empty? ...
%]   
%]
%\end{verbatim}	
%\end{frame}
%
%\begin{frame}[fragile]
%\begin{verbatim}
%to-report search-before-buying
%let cheapest-price 100000
%let cheapest-route []
%
%repeat n-scans [
%let this-route []
%let cost 0
%let to-buy shopping-list
%let visited []
%while [ not empty? to-buy] [
%  let stall one-of traders with [ not member? self visited ]
%  if stall = nobody [ show (word "Trying to buy " to-buy ", 
%       but no trader sells it.")
%    set shopping-list []
%    report []
%  ]
%\end{verbatim}	
%\end{frame}
%
%\begin{frame}[fragile]
%\begin{verbatim}
%  set visited lput stall visited
%  let purchases buy-from-stall to-buy stall
%  if not empty? purchases [
%    set this-route lput stall this-route
%    foreach purchases [ ?1 ->
%      set cost cost + produce-price ?1 stall
%      set to-buy remove ?1 to-buy ]
%  ]
%]
%if cost < cheapest-price [
%  set cheapest-price cost
%  set cheapest-route this-route
%]
%]
%report cheapest-route
%end	
%\end{verbatim}	
%\end{frame}
%
%\begin{frame}[fragile]
%\begin{verbatim}
%to-report produce-price [ produce stall ]
%report item (position produce [stock] of stall) 
%[ prices ] of stall
%end
%
%to-report buy-from-stall [ what-to-buy stall ]
%report filter [ ?1 -> member? ?1 [stock] of stall ] 
%what-to-buy
%end	
%\end{verbatim}	
%\end{frame}
%
%\begin{frame}[fragile]
%\begin{verbatim}
%to run-experiment
%set n-scans 1
%let runs-per-trial 100
%while [ n-scans <= 10 ] [
%let total-of-averages 0
%repeat runs-per-trial [
%  ask shoppers [
%    set not-yet-visited traders
%    set shopping-list n-of (1 + random 8) 
%      fruit-and-veg
%    set spent 0
%  ]
%  go
%  set total-of-averages total-of-averages 
%   + mean [ spent ] of shoppers
%]
%\end{verbatim}	
%\end{frame}
%
%\begin{frame}[fragile]
%\begin{verbatim}
%show (word "Mean of average of cost of shopping lists over " 
% runs-per-trial " runs for " n-scans 
% " scans = " (total-of-averages / runs-per-trial))
%set n-scans n-scans + 1
%]
%show "Finished."
%end
%\end{verbatim}	
%\end{frame}

%\begin{frame}
%\bibliographystyle{apalike}	
%\end{frame}

\bibliography{my_library}

\end{document}